\documentclass[11pt,a4paper]{article}

\usepackage[margin=1in]{geometry}
\usepackage{amsmath,amssymb}
\usepackage{mathtools}
\usepackage{enumitem}
\usepackage{hyperref}

\setlist[itemize]{itemsep=2pt, topsep=4pt}

\title{COMP3204 --- Lecture 3 Notes: Image Sampling (DFT, FFT, Sampling, Aliasing)}
\author{}
\date{}

\begin{document}
\maketitle

\section{Why frequency space matters in Computer Vision}
\begin{itemize}
  \item An image can be treated as a \textbf{2D signal}: intensity varies over \textbf{space} (pixel coordinates).
  \item The Fourier viewpoint: represent a signal as a sum of \textbf{sinusoids} at different \textbf{frequencies}.
  \item In images, frequency means \textbf{spatial frequency}:
  \begin{itemize}
    \item \textbf{Low spatial frequency}: intensity changes slowly (smooth shading, broad structure).
    \item \textbf{High spatial frequency}: intensity changes rapidly (edges, fine texture, noise).
  \end{itemize}
  \item Why it is useful (as stated in the lecture material):
  \begin{itemize}
    \item \textbf{Understanding/analysis} of image content at different scales.
    \item \textbf{Speeding up computation} (many operations become efficient using FFT).
    \item \textbf{Representation properties} (e.g., shift effects become simple).
    \item \textbf{Coding/compression} (store/keep important components).
    \item \textbf{Recognition/understanding} (e.g., texture).
  \end{itemize}
\end{itemize}

\section{1D Discrete Fourier Transform (DFT)}
\subsection{Forward 1D DFT}
\begin{itemize}
  \item Input: $p[i]$ for $i=0,\dots,N-1$ (a sampled 1D signal).
  \item Output: complex frequency coefficients $Fp[u]$ for $u=0,\dots,N-1$.
  \item Definition:
  \begin{equation}
    Fp[u] \;=\; \frac{1}{N}\sum_{i=0}^{N-1} p[i]\;e^{-j\frac{2\pi}{N}ui}.
  \end{equation}
  \item Meaning of symbols:
  \begin{itemize}
    \item $i$: sample index (position in the signal).
    \item $u$: frequency index (which discrete sinusoid is being measured).
    \item $j$: imaginary unit, $j^2=-1$.
    \item $e^{-j\theta}=\cos\theta - j\sin\theta$: complex sinusoid basis.
  \end{itemize}
  \item The $1/N$ scaling is chosen so that the \textbf{DC term} becomes the \textbf{average} of the samples.
\end{itemize}

\subsection{Magnitude and phase: what the complex output means}
\begin{itemize}
  \item Each $Fp[u]$ is complex and can be written as:
  \[
    Fp[u] = \operatorname{Re}(Fp[u]) + j\,\operatorname{Im}(Fp[u]).
  \]
  \item \textbf{Magnitude} (strength of that frequency):
  \[
    |Fp[u]|=\sqrt{\operatorname{Re}(Fp[u])^2+\operatorname{Im}(Fp[u])^2}.
  \]
  \item \textbf{Phase} (alignment/shift information):
  \[
    \arg(Fp[u])=\tan^{-1}\!\left(\frac{\operatorname{Im}(Fp[u])}{\operatorname{Re}(Fp[u])}\right).
  \]
  \item DC component:
  \begin{itemize}
    \item For $u=0$, $e^{-j(0)}=1$, so
    \[
      Fp[0]=\frac{1}{N}\sum_{i=0}^{N-1} p[i],
    \]
    which is the \textbf{mean value} of the signal.
  \end{itemize}
\end{itemize}

\subsection{Inverse 1D DFT (reconstruction)}
\begin{itemize}
  \item You can reconstruct the original samples from all frequency coefficients:
  \begin{equation}
    p[i] \;=\; \sum_{u=0}^{N-1} Fp[u]\;e^{+j\frac{2\pi}{N}ui}.
  \end{equation}
  \item Interpretation: the signal is rebuilt by adding all sinusoidal components (including DC).
\end{itemize}

\section{Transform-pair intuition (pulse example)}
\begin{itemize}
  \item A short, sharp pulse in the sample domain contains many frequencies.
  \item Key exam takeaway:
  \begin{itemize}
    \item \textbf{Sharp changes} in a signal require \textbf{high-frequency} components.
    \item In images, \textbf{edges} are high-frequency phenomena.
  \end{itemize}
\end{itemize}

\section{Reconstruction using only some frequencies}
\begin{itemize}
  \item If you reconstruct using only:
  \begin{itemize}
    \item \textbf{Low frequencies}: you keep coarse structure but lose detail (blurry/smooth).
    \item \textbf{More frequencies}: you progressively recover sharper details.
  \end{itemize}
  \item Conceptual message from the lecture visuals:
  \begin{itemize}
    \item Low frequencies carry global structure.
    \item High frequencies add fine detail and sharpness.
  \end{itemize}
\end{itemize}

\section{2D DFT for images}
\subsection{Forward 2D DFT}
\begin{itemize}
  \item Image samples: $P[x,y]$ for $x,y=0,\dots,N-1$.
  \item Frequency coefficients: $FP[u,v]$, where $u$ and $v$ are horizontal/vertical spatial frequency indices.
  \item Definition:
  \begin{equation}
    FP[u,v] \;=\; \frac{1}{N^2}\sum_{y=0}^{N-1}\sum_{x=0}^{N-1}
    P[x,y]\;e^{-j\frac{2\pi}{N}(ux+vy)}.
  \end{equation}
  \item DC component $FP[0,0]$ equals the \textbf{average intensity} of the entire image.
\end{itemize}

\subsection{Inverse 2D DFT}
\begin{itemize}
  \item Reconstruction:
  \begin{equation}
    P[x,y] \;=\; \sum_{v=0}^{N-1}\sum_{u=0}^{N-1} FP[u,v]\;e^{+j\frac{2\pi}{N}(ux+vy)}.
  \end{equation}
  \item Interpretation: the image is a sum of 2D sinusoidal ``patterns'' (basis functions).
\end{itemize}

\subsection{Visualising the Fourier transform}
\begin{itemize}
  \item In practice, displays often show:
  \begin{itemize}
    \item \textbf{Magnitude} (often using a log scale so the large DC component does not dominate).
    \item \textbf{Phase} (contains crucial alignment/structure information even if it looks unintuitive).
  \end{itemize}
  \item Keeping only low-frequency coefficients reconstructs a recognizable but blurry image; adding higher frequencies restores detail.
\end{itemize}

\section{FFT: fast computation of the DFT}
\begin{itemize}
  \item Direct DFT computation is expensive because you compute a sum for each frequency coefficient.
  \item \textbf{FFT (Fast Fourier Transform)} computes the same DFT much faster.
  \item Core algorithmic idea (as in the lecture material):
  \begin{itemize}
    \item \textbf{Divide-and-conquer}: split into smaller DFTs, then combine results efficiently (often visualised as repeated add/subtract ``butterfly'' patterns).
  \end{itemize}
  \item Why it matters in vision: makes frequency-domain processing feasible on real image sizes.
\end{itemize}

\section{Fourier properties used in vision}
\subsection{Shift effect (magnitude invariance)}
\begin{itemize}
  \item Shifting a signal/image in space changes \textbf{phase} but preserves \textbf{magnitude}.
  \item 1D property (conceptual form):
  \[
    \mathcal{F}\{p(t-\tau)\} = e^{-j\omega\tau}P(\omega).
  \]
  \item Since $|e^{-j\omega\tau}|=1$, the magnitude is unchanged:
  \[
    |\mathcal{F}\{p(t-\tau)\}| = |P(\omega)|.
  \]
  \item Exam implication:
  \begin{itemize}
    \item Magnitude-based features can be robust to translation.
    \item Phase carries location/alignment information.
  \end{itemize}
\end{itemize}

\subsection{Rotation}
\begin{itemize}
  \item Rotating an image rotates its Fourier transform by the same angle.
  \item Intuition: orientation of structures (e.g., stripes/edges) corresponds to oriented energy in frequency space.
\end{itemize}

\section{Filtering using the Fourier transform}
\begin{itemize}
  \item Filtering idea: modify the frequency coefficients, then apply the inverse transform.
  \item \textbf{Low-pass filtering}:
  \begin{itemize}
    \item Keep low frequencies, suppress high frequencies.
    \item Effect: smoothing / blur, reduces fine detail and noise.
  \end{itemize}
  \item \textbf{High-pass filtering}:
  \begin{itemize}
    \item Suppress low frequencies, keep high frequencies.
    \item Effect: emphasizes edges and fine details.
  \end{itemize}
\end{itemize}

\section{Sampling theory}
\subsection{What sampling is}
\begin{itemize}
  \item Sampling converts a continuous signal into discrete measurements.
  \item In images: the sensor/ADC measures intensity on a discrete pixel grid.
  \item Good sampling: enough samples to capture the fastest variations you care about.
  \item Poor sampling: too few samples can make high-frequency content appear as a different (lower-frequency) pattern.
\end{itemize}

\subsection{Sampling in frequency domain: spectrum replication idea}
\begin{itemize}
  \item Lecture key relation (stated conceptually):
  \[
    \mathcal{F}\{x(t)\,\delta(t)\} = \mathcal{F}\{x(t)\} * \mathcal{F}\{\delta(t)\}.
  \]
  \item Takeaway used in sampling theory:
  \begin{itemize}
    \item Sampling causes \textbf{copies} of the spectrum to repeat in frequency.
  \end{itemize}
\end{itemize}

\subsection{Nyquist sampling theorem}
\begin{itemize}
  \item To reconstruct a signal from samples:
  \[
    f_s \ge 2f_{\max},
  \]
  where $f_s$ is the sampling frequency and $f_{\max}$ is the highest frequency present in the original signal.
  \item Frequency-domain interpretation:
  \begin{itemize}
    \item If $f_s$ is high enough, repeated spectra do \textbf{not overlap} $\Rightarrow$ reconstruction is possible.
    \item If $f_s < 2f_{\max}$, the repeated spectra \textbf{overlap} $\Rightarrow$ aliasing occurs.
  \end{itemize}
  \item Practical guideline from the slides: ``two pixels for every pixel of interest'' (sample densely enough to represent the smallest detail you care about).
\end{itemize}

\section{Aliasing}
\begin{itemize}
  \item \textbf{Aliasing} happens when sampling is too low, causing high-frequency content to appear as lower frequency content (false patterns).
  \item Spatial aliasing example (as shown): fine repetitive textures (e.g., blinds) can produce large-scale incorrect patterns when downsampled.
  \item Temporal aliasing (wagon-wheel effect):
  \begin{itemize}
    \item Sampling in time is the frame rate.
    \item If motion changes too much between frames, a wheel can appear to rotate backwards or stand still.
  \end{itemize}
  \item Conceptual avoidance (as implied by the sampling theorem material):
  \begin{itemize}
    \item Increase sampling rate (higher resolution / higher frame rate), and/or
    \item Remove high frequencies before sampling (pre-filtering / anti-aliasing).
  \end{itemize}
\end{itemize}

\section{Mentioned (not expanded in this lecture)}
\begin{itemize}
  \item Compressed sensing
  \item Sparsity
  \item Regularisation
\end{itemize}

\end{document}
